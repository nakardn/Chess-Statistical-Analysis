\documentclass[conference]{IEEEtran}
\usepackage[table]{xcolor}
\usepackage{ifthen, xkeyval, tikz, calc}


\usepackage{hyperref}
\usepackage{url}
\usepackage{graphicx}
% \usepackage{svg}
\usepackage{subcaption}
\usepackage{float}
\usepackage{tikz}
\usepackage{lipsum}
\usepackage{booktabs}
\usepackage{listings}
\lstset{
  basicstyle=\ttfamily,
  columns=fullflexible,
  frame=single,
  breaklines=true,
  postbreak=\mbox{\textcolor{red}{$\hookrightarrow$}\space},
}
\usepackage{pifont}% http://ctan.org/pkg/pifont
\newcommand{\cmark}{\checkmark}%
\newcommand{\xmark}{\ding{55}}%


\graphicspath{{./figures/}}
%\graphicspath{{./tmp_figures/}}

% \let\marginpar\oldmarginpar
\setlength{\marginparwidth}{5cm}
% \newcommand{\Bnote}[1]{\textbf{[Boaz: #1]}}
\usepackage{xargs}                      % Use more than one optional parameter in a new commands
\usepackage[colorinlistoftodos,prependcaption, textwidth=3cm]{todonotes}



\title{
Statistical Theory\\
Chess Dataset Analysis
}

% Authors must not appear in the submitted version. They should be hidden
% as long as the \iclrfinalcopy macro remains commented out below.
% Non-anonymous submissions will be rejected without review.

\author{
   \IEEEauthorblockN{Dor Boker, Itamar Nakar}
   \IEEEauthorblockA{
      I.D: , 325829000\\
      Email: , itamar.nakar@gmail.com
   }
}

\newcommand{\fix}{\marginpar{FIX}}
\newcommand{\new}{\marginpar{NEW}}

\listfiles
\begin{document}


\maketitle

\section{Introduction}
Chess is one of humanity's oldest board games. It is played by two players, nicknamed White and Black over an $8\times8$ grid, where each players controls 16 pieces which correspond to its own color. The Goal of each player is to capture the opponents King piece(or, more accurately, to make it so the opponent's King cannot escape capture). Chess doesn't involve any luck, no hidden information. It is determined by the players' knowlage of the game, their strategical and analytical capabilities.

The length of a chess match can be influenced by a variety of factors, including player skill level, strategy, and in-game dynamics. In this project we investigates the relationship between player ratings and the duration of chess matches, seeking to determine whether higher-rated players tend to play shorter or longer games. In addition to player ratings, we explore other game-related variables that might impact match length, such as opening moves, the type of result (checkmate, resignation, or draw) and more.

Using a dataset of chess games, we apply statistical methods such as correlation analysis and regression models to analyze the influence of these factors. The study aims to offer insights into how skill level and game dynamics affect the length of a match, contributing to a broader understanding of player performance in chess.


\end{document}
